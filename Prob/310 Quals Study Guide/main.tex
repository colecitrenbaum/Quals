\documentclass{article}
\usepackage{amsmath, amssymb}
\usepackage{hw}
\usepackage{tcolorbox}
\usepackage[ruled,vlined]{algorithm2e}

% ------------------  Bibliography  ------------------
%\usepackage[
%  backend=biber,
%  style=numeric,      % plain numbers: [1]
%  sorting=none,       % keep refs in citation order
%  maxbibnames=99      % show all authors in bib
%]{biblatex}
\usepackage{booktabs}
% ----------------------------------------------------

\tcbuselibrary{breakable}
\title{310 Quals Strategy Compendium}
\newcommand{\answer}[1]{%
  \begin{tcolorbox}[
    colback=gray!9.8,
    boxrule=0.5pt,
    breakable]
  \small #1
  \end{tcolorbox}}
\newcommand\myworries[1]{\textcolor{red}{#1}}
\newcommand{\simiid}{\overset{iid}\sim }
\newcommand{\rank}{\operatorname{rank}}
\newcommand{\range}{\operatorname{range}}
\newcommand{\row}{\operatorname{row}}
\newcommand{\Proj}{\operatorname{Proj}}

\begin{document}
\maketitle

\section{Permutation and counting facts}
\begin{fact}[Number derangements of $k$-element set]
Derangements: $D_n$ is the number of permutations with no fixed points.\\

Via inclusion exclusion.
$$D_k = k! \sum_{j=0}^{k} \frac{(-1)^j}{j!}.$$
Eg, if $T$ is number of fixed points:
$$P(T=k) = \frac{1}{n!} \binom{n}{k} D_{n-k},$$
since the remaining $n-k$ must \textbf{not} be fixed.

    
\end{fact}
Apply the above to "Distribution of number of fixed points"- type questions. 

\begin{fact}[Catalan Numbers]
	Catalan Numbers: 
	
	$$C_n = \frac{1}{n+1} \binom{2n}{n} = \frac{(2n)!}{(n+1)! n!} \quad n\geq 0$$
	
	
		Dyck Paths 
\end{fact}

\begin{definition}[Cycles]
	Cycle of a permutations
\end{definition}

\begin{definition}[Descents]
	
\end{definition}

Reference: check Persi and Susan's paper


\section{Distribution Facts}
\subsection{Gaussian Facts}
\begin{fact}[Max of Gaussians and fluctuations]
	Max of Gaussians $\sqrt{2\log n}$ with fluctuations $1/\sqrt{\log n} $
\end{fact}
\begin{fact}[Max of sub-gaussian] 
	Expectation upper bound applies to correlated sub-gaussian reandom variables -- see Vershynin. 
\end{fact}
\begin{fact}[Mill's Ratio]
	
\end{fact}

\begin{fact}[Normal Conditional Distributions]
	$$(X,Y) \sim MVN(\mu, \Sigma \implies X|Y \sim \ldots $$
\end{fact}

\subsection{Poisson, Exponential Distribution}
Superposition and thinning.

\begin{fact}[Superposition]
	See Poisson with integer mean? Try superposition:
	
	$$Pois(n) = \sum_{i=1}^n Pois(1)$$
\end{fact}

\begin{fact}[Renyi Representation of Exponential]
	
\end{fact}
\begin{fact}[Maximum, minimum of Exponential]
	
\end{fact}


\section{Stein's Method (Poisson)}


\subsection{Method 1 - Dependency Graphs}

\subsection{Method 2 - when dependency graph doesn't work (ie complete)}

\begin{example}[Fixed Points - 310a HW8]
    Let $\sigma$ be a uniformly chosen permutation in the symmetric group $S_n$. Let $W = \#\{i : \sigma(i) = i\}$ (the number of fixed points in $\sigma$). Show that $W$ has an approximate Poisson(1) distribution by using Stein's method to get an upper bound on $\|P_W - \text{Poisson}(1)\|$. (Hint: see section 4.5 of Arratia-Goldstein-Gordon.) Give details for this specific case.

\answer{
Let $I = [n]$. We choose $B_\alpha = \{\alpha\}$ and use Theorem 1 from Arratia-Goldstein-Gordon. 

For each $i\in I$, let

$$X_i = \begin{cases}
    1 \text{ if } \sigma(i) = i\\
    0 \text{ otherwise}
\end{cases}.$$
Naturally, $P(X_i =1) = \frac{1}{n}$. We let $W = \sum_{i\in I} X_i$ and $\lambda = E[W] = 1$. We now use Stein's method as given in Arratia-Goldstein-Gordon Theorem 1 to get an upper bound on $\lVert P_W - Pois(1)\rVert$.
$$b_1 = \sum_{\alpha\in I} \sum_{\beta \in B_\alpha} p_\alpha p_\beta = \sum_{\alpha \in I} p_\alpha^2 = \frac{1}{n}.$$
Next, because we let $B_\alpha = \{\alpha\}$, 
$$b_2 = 0.$$
Finally, for the third term, by Lemma 2 (p 418) in Arratia et al,
$$b_3 \leq \min_{1<k<n} (\frac{2k}{n-k} + 2n 2^{-k} e^{e})\sim 2\frac{(2log_2 n + e/ln 2)}{n},$$
due to the fact that $\lambda=1$ in our problem, so $\lambda = o(n)$

Now note that as $n\to \infty$, $b_1\to 0$ and $b_3\to 0$, so, noting that the Arratia paper's definition of TV distance is twice our definition of TV distance:

$$\lVert P_W - Pois(1)\rVert\leq b_1 + b_2 + b_3 = \frac{1}{n} + \frac{4\log_2 n + 2\epsilon/\ln 2}{n} + o(1)$$
Now as $n\to \infty$, $b_1 \to 0$ and $b_3 \to 0$, so $\lVert P_W - Pois(1)\rVert \to 0$. 
}
\end{example}

\begin{example}[Near Fixed Points- 2004 Q2]
    
\end{example}



\newpage 
\section{Approximations}
$$1-x \leq e^{-x} \quad 1-x\geq e^{-2x} \quad \text{both for small } x?$$  


\end{document}