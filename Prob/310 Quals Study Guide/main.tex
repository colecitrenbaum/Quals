\documentclass{article}
\usepackage{amsmath, amssymb}
\usepackage{hw}
\usepackage{tcolorbox}
\usepackage[ruled,vlined]{algorithm2e}

% ------------------  Bibliography  ------------------
%\usepackage[
%  backend=biber,
%  style=numeric,      % plain numbers: [1]
%  sorting=none,       % keep refs in citation order
%  maxbibnames=99      % show all authors in bib
%]{biblatex}
\usepackage{booktabs}
% ----------------------------------------------------

\tcbuselibrary{breakable}
\title{310 Quals Strategy Compendium}
\newcommand{\answer}[1]{%
  \begin{tcolorbox}[
    colback=gray!9.8,
    boxrule=0.5pt,
    breakable]
  \small #1
  \end{tcolorbox}}
\newcommand\myworries[1]{\textcolor{red}{#1}}
\newcommand{\simiid}{\overset{iid}\sim }
\newcommand{\rank}{\operatorname{rank}}
\newcommand{\range}{\operatorname{range}}
\newcommand{\row}{\operatorname{row}}
\newcommand{\Proj}{\operatorname{Proj}}

\begin{document}
\maketitle

\section{Permutation and counting facts}
\begin{fact}[Number derangements of $k$-element set]
Derangements: $D_n$ is the number of permutations with no fixed points.\\

Via inclusion exclusion.
$$D_k = k! \sum_{j=0}^{k} \frac{(-1)^j}{j!}.$$
Eg, if $T$ is number of fixed points:
$$P(T=k) = \frac{1}{n!} \binom{n}{k} D_{n-k},$$
since the remaining $n-k$ must \textbf{not} be fixed.

    
\end{fact}
Apply the above to "Distribution of number of fixed points"- type questions. 

\begin{fact}[Catalan Numbers]
	Catalan Numbers: 
	
	$$C_n = \frac{1}{n+1} \binom{2n}{n} = \frac{(2n)!}{(n+1)! n!} \quad n\geq 0$$
	
	
		Dyck Paths 
\end{fact}

\begin{definition}[Cycles]
	Cycle of a permutations
\end{definition}

\begin{definition}[Descents]
	
\end{definition}

Reference: check Persi and Susan's paper


\section{Distribution Facts}
\subsection{Gaussian Facts}
\begin{fact}[Max of Gaussians and fluctuations]
	Max of Gaussians $\sqrt{2\log n}$ with fluctuations $1/\sqrt{\log n} $
\end{fact}
\begin{fact}[Max of sub-gaussian] 
	Expectation upper bound applies to correlated sub-gaussian reandom variables -- see Vershynin. 
\end{fact}
\begin{fact}[Mill's Ratio]
	
\end{fact}

\begin{fact}[Normal Conditional Distributions]
	$$(X,Y) \sim MVN(\mu, \Sigma \implies X|Y \sim \ldots $$
\end{fact}

\subsection{Poisson, Exponential Distribution}
Superposition and thinning.

\begin{fact}[Superposition]
	See Poisson with integer mean? Try superposition:
	
	$$Pois(n) = \sum_{i=1}^n Pois(1)$$
\end{fact}

\begin{fact}[Renyi Representation of Exponential]
	
\end{fact}
\begin{fact}[Maximum, minimum of Exponential]
	
\end{fact}


\section{Basic set theory and measure theory}
\begin{definition}[Sigma Algebra]
\end{definition}
\begin{definition}[Algebra/Field]
\end{definition}





\begin{definition}[Outer measure]
Defined by 
\begin{enumerate}
	\item A non-negative set function 
	\item ......
\end{enumerate}
The typical outer measure is 
\myworries{WRite this out}
\end{definition}

\begin{definition}[Measurable sets]
$E$ is $\mu^*$ measurable if for all $B\subset \Omega$:

$$\mu^* (B) = \mu^*(B\cap A) + \mu^*(B \cap A^c).$$
\end{definition}

Littlewood's principles idea. 

\section{Pi-Lambda and Good Sets}

\begin{definition}[$\pi$-system]
Collection of sets that is closed under \textbf{finite intersections}
\end{definition}
\begin{definition}[$\lambda$-system]
$L$ lambda system if
\begin{enumerate}
	\item $\Omega \in L$
	\item closed under complements
	\item closed under countable disjoint unions
\end{enumerate}
Alternative definition:
\begin{enumerate}
	\item $\Omega \in L$
	\item $A,B \in L$ and $A\subset B$ then $B\setminus A \in L$
	\item $A_1, A_2,\ldots, \in L$ an increasing sequence of sets, then $\bigcup_i A_i \in L$
\end{enumerate}
\end{definition}

\begin{theorem}[$\pi-\lambda$ Theorem]
If $P$ is a $\pi$ system and $L$ is a $\lambda$-system with $P\subset L$, then $\sigma(P)\subset L$.\\
Use to proof uniqueness of extension from an algebra to the sigma field.
\end{theorem}
\begin{example}[Quals 2017, Question 2]
\end{example}


\begin{theorem}[Monotone Class Theorem]
Use def:
\begin{definition}[Monotone Class]
$M$ is a monotone class if 
\begin{enumerate}
	\item Closed under increasing unions
	\item Closed under decreasing intersections
\end{enumerate}
\end{definition}
If an algebra $A$ is contained in a monotone class $M$, then $\sigma(A) \subset \sigma(M)$.
\end{theorem}
\begin{theorem}[Monotone Class for functions]
\myworries{Double check this}\\
$M$ be a vector space of measurable functions such that 
\begin{enumerate}
	\item $1 \in M$
	\item $M^+$ (positive functions in $M$) is closed under inreasing limits 
	\item $\Ind_A \in M$ for all $A$ in a pi system generating $\calF$.
\end{enumerate}
Then $M$ contains all bounded measurable functions
\end{theorem}



\begin{theorem}[Independent $\pi$ systems generate independent sigma algebras]
$$\{C_i\}_{i\in I} \text { independent} \implies \{\sigma(C_i)\} \text{ independent} $$
\begin{example}[Prove two random variables independent]
Check that generating pi systems are independent, ie $P(X\leq b, Y\leq a) = \prod P(X\leq b) P(Y\leq a)$. 
\end{example}

\end{theorem}

\section{Borel Cantelli} 
\section{Integration}
\begin{definition}[Lebesgue Integral]
$$\int_\Omega f(\omega) d\mu(\omega) = \sup \left (\sum_{i=1}^n \nu_i \mu(A_i)\right)$$
where $\nu_i = \inf_{\omega \in A_i} f(\omega)$ and the sup is over all partitions of $\Omega$.
\end{definition}
\begin{theorem}[MCT]
Note that can also do for general functions (not necessarily non negative) so long as $f_n \geq g \in L^1$.
\end{theorem}
\begin{theorem}[Fatou's Lemma]
$$\int \lim \inf_n f_n d\mu \leq \lim \inf_n \int f_n d\mu $$
\end{theorem}
\begin{theorem}[DCT]
If $f_n \to f$ a.e $\omega$, and there exists $g$ such that $|f_n(\omega)|\leq g(\omega)$ a.e $\omega$ and $\int gd\mu <\infty$ then exchange integral and limit. Must dominate the sequence. 
\end{theorem}

\begin{theorem}[Scheffe's Lemma]
Is a statement about combining $L^1$ and as convergence. If $f_n \conas f\in L^1$, then:
$$\| f_n -f\|_{L^1} \to 0 \iff \int |f_n| d\mu \to \int |f|d\mu $$
\end{theorem}


\begin{theorem}[Generalized DCT]
\myworries{????}
If $|f_n|\leq g_n$ such that $g_n \to g \in L^1$ (convergence in $L^1$) then $\int f_n d\mu \to \int fd\mu$
\end{theorem}

\begin{theorem}[Reverse Fatou]
If $f_n \leq g \in L^1$ then :

$$\lim \sup_n \int f_n d\mu \leq \int \lim \sup f_n d\mu$$
\end{theorem}

\section{Product $\sigma$-algebras}

\section{Stein's Method (Poisson)}


\subsection{Method 1 - Dependency Graphs}

\subsection{Method 2 - when dependency graph doesn't work (ie complete)}

\begin{example}[Fixed Points - 310a HW8]
    Let $\sigma$ be a uniformly chosen permutation in the symmetric group $S_n$. Let $W = \#\{i : \sigma(i) = i\}$ (the number of fixed points in $\sigma$). Show that $W$ has an approximate Poisson(1) distribution by using Stein's method to get an upper bound on $\|P_W - \text{Poisson}(1)\|$. (Hint: see section 4.5 of Arratia-Goldstein-Gordon.) Give details for this specific case.

\answer{
Let $I = [n]$. We choose $B_\alpha = \{\alpha\}$ and use Theorem 1 from Arratia-Goldstein-Gordon. 

For each $i\in I$, let

$$X_i = \begin{cases}
    1 \text{ if } \sigma(i) = i\\
    0 \text{ otherwise}
\end{cases}.$$
Naturally, $P(X_i =1) = \frac{1}{n}$. We let $W = \sum_{i\in I} X_i$ and $\lambda = E[W] = 1$. We now use Stein's method as given in Arratia-Goldstein-Gordon Theorem 1 to get an upper bound on $\lVert P_W - Pois(1)\rVert$.
$$b_1 = \sum_{\alpha\in I} \sum_{\beta \in B_\alpha} p_\alpha p_\beta = \sum_{\alpha \in I} p_\alpha^2 = \frac{1}{n}.$$
Next, because we let $B_\alpha = \{\alpha\}$, 
$$b_2 = 0.$$
Finally, for the third term, by Lemma 2 (p 418) in Arratia et al,
$$b_3 \leq \min_{1<k<n} (\frac{2k}{n-k} + 2n 2^{-k} e^{e})\sim 2\frac{(2log_2 n + e/ln 2)}{n},$$
due to the fact that $\lambda=1$ in our problem, so $\lambda = o(n)$

Now note that as $n\to \infty$, $b_1\to 0$ and $b_3\to 0$, so, noting that the Arratia paper's definition of TV distance is twice our definition of TV distance:

$$\lVert P_W - Pois(1)\rVert\leq b_1 + b_2 + b_3 = \frac{1}{n} + \frac{4\log_2 n + 2\epsilon/\ln 2}{n} + o(1)$$
Now as $n\to \infty$, $b_1 \to 0$ and $b_3 \to 0$, so $\lVert P_W - Pois(1)\rVert \to 0$. 
}
\end{example}

\begin{example}[Near Fixed Points- 2004 Q2]
    
\end{example}



\newpage 
\section{Approximations}
$$1-x \leq e^{-x} \quad 1-x\geq e^{-2x} \quad \text{both for small } x?$$  


\end{document}