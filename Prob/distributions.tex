\documentclass{article}
\usepackage{amsmath, amssymb}
\usepackage{hw}
\usepackage{tcolorbox}
\usepackage[ruled,vlined]{algorithm2e}

% ------------------  Bibliography  ------------------
%\usepackage[
%  backend=biber,
%  style=numeric,      % plain numbers: [1]
%  sorting=none,       % keep refs in citation order
%  maxbibnames=99      % show all authors in bib
%]{biblatex}
\usepackage{booktabs}
% ----------------------------------------------------

\tcbuselibrary{breakable}
\title{Distribution stuff comnpendium}
\newcommand{\answer}[1]{%
  \begin{tcolorbox}[
    colback=gray!9.8,
    boxrule=0.5pt,
    breakable]
  \small #1
  \end{tcolorbox}}
\newcommand\myworries[1]{\textcolor{red}{#1}}
\newcommand{\simiid}{\overset{iid}\sim }
\newcommand{\rank}{\operatorname{rank}}
\newcommand{\range}{\operatorname{range}}
\newcommand{\row}{\operatorname{row}}
\newcommand{\Proj}{\operatorname{Proj}}

\begin{document}
\maketitle
% ------------------------------------------------------------------
%  Discrete distributions
% ------------------------------------------------------------------
\begin{table}[ht]
\centering
\renewcommand{\arraystretch}{1.3}
\small
\begin{tabular}{@{}l l l l@{}}
\toprule
\textbf{Distribution} & \textbf{Support} & \textbf{p.m.f.\ $P(X=x)$} & \textbf{c.d.f.\ $F(x)$} \\
\midrule
Bernoulli$(p)$            & $\{0,1\}$ &
$ p^{x}(1-p)^{1-x}$ &
$ \mathbf 1_{x\ge 1}p + \mathbf 1_{0\le x<1}(1-p)$ \\[3pt]

Binomial$(n,p)$           & $\{0,\dots,n\}$ &
$ \displaystyle\binom{n}{x}p^{x}(1-p)^{n-x}$ &
$ \displaystyle\sum_{k=0}^{\lfloor x\rfloor}\binom{n}{k}p^{k}(1-p)^{\,n-k}$ \\[6pt]

Geometric\,$(p)$\,(shift-1) & $\{1,2,\dots\}$ &
$ (1-p)^{x-1}p$ &
$ 1-(1-p)^{\lfloor x\rfloor}$ \\[3pt]

Negative Binomial$(r,p)$ &
$\{0,1,\dots\}$ &
$ \displaystyle\binom{r+x-1}{x}(1-p)^{r}\,p^{x}$ &
$ \displaystyle 1-\mathrm{B}_{p}\!\!\bigl(\lfloor x\rfloor+1,r\bigr)$\quad($\mathrm{B}_{p}$ = regularized Beta) \\[6pt]

Poisson$(\lambda)$        & $\{0,1,\dots\}$ &
$ e^{-\lambda}\dfrac{\lambda^{x}}{x!}$ &
$ e^{-\lambda}\displaystyle\sum_{k=0}^{\lfloor x\rfloor}\dfrac{\lambda^{k}}{k!}$ \\[6pt]

Hypergeometric$(N,K,n)$   & $\{0,\dots,n\}$ &
$ \displaystyle\frac{\binom{K}{x}\binom{N-K}{\,n-x\,}}{\binom{N}{n}}$ &
$ \displaystyle\sum_{k=0}^{\lfloor x\rfloor}\frac{\binom{K}{k}\binom{N-K}{\,n-k\,}}{\binom{N}{n}}$ \\[6pt]

Discrete Uniform$(a{:}b)$ & $\{a,a\!+\!1,\dots,b\}$ &
$ \dfrac{1}{b-a+1}$ &
$ \dfrac{\lfloor x\rfloor-a+1}{b-a+1}\,\,\mathbf 1_{x\ge a}$ \\ 
\bottomrule
\end{tabular}
\caption{Common \emph{discrete} distributions.  Here $\mathbf 1_{A}$ is the indicator of event \(A\).}
\end{table}

% ------------------------------------------------------------------
%  Continuous distributions
% ------------------------------------------------------------------
\begin{table}[ht]
\centering
\renewcommand{\arraystretch}{1.3}
\small
\begin{tabular}{@{}l l l l@{}}
\toprule
\textbf{Distribution} & \textbf{Support} & \textbf{p.d.f.\ $f(x)$} & \textbf{c.d.f.\ $F(x)$} \\
\midrule
Uniform$(a,b)$            & $(a,b)$ &
$ \dfrac{1}{b-a}$ &
$ \dfrac{x-a}{b-a}$,\ $a<x<b$ \\[6pt]

Exponential$(\lambda)$    & $(0,\infty)$ &
$ \lambda e^{-\lambda x}$ &
$ 1-e^{-\lambda x}$ \\[6pt]

Gamma$(\alpha,\theta)$    & $(0,\infty)$ &
$ \dfrac{x^{\alpha-1}e^{-x/\theta}}{\Gamma(\alpha)\theta^{\alpha}}$ &
$ \dfrac{\gamma(\alpha,x/\theta)}{\Gamma(\alpha)}$\quad
($\gamma$ = lower incomplete $\Gamma$) \\[6pt]

$\chi^{2}_{k}$            & $(0,\infty)$ &
$ \dfrac{x^{k/2-1}e^{-x/2}}{2^{k/2}\Gamma(k/2)}$ &
$ P\!\bigl(k/2,x/2\bigr)$ (regularized $\Gamma$) \\[6pt]

Normal$(\mu,\sigma^{2})$  & $(-\infty,\infty)$ &
$ \dfrac{1}{\sqrt{2\pi\sigma^{2}}}\exp\!\Bigl[-\dfrac{(x-\mu)^{2}}{2\sigma^{2}}\Bigr]$ &
$ \Phi\!\Bigl(\dfrac{x-\mu}{\sigma}\Bigr)$\quad($\Phi$ = standard normal c.d.f.) \\[9pt]

Lognormal$(\mu,\sigma)$   & $(0,\infty)$ &
$ \dfrac{1}{x\sigma\sqrt{2\pi}}\exp\!\Bigl[-\dfrac{(\ln x-\mu)^{2}}{2\sigma^{2}}\Bigr]$ &
$ \Phi\!\Bigl(\dfrac{\ln x-\mu}{\sigma}\Bigr)$ \\[9pt]

Student-$t(\nu)$          & $(-\infty,\infty)$ &
$ \dfrac{\Gamma\!\bigl(\tfrac{\nu+1}{2}\bigr)}
         {\sqrt{\nu\pi}\,\Gamma\!\bigl(\tfrac{\nu}{2}\bigr)}
         \bigl(1+\tfrac{x^{2}}{\nu}\bigr)^{-(\nu+1)/2}$ &
$ \tfrac12 + x\,\dfrac{\,{}_2F_{1}\!\bigl(\tfrac12, \tfrac{\nu+1}{2}; \tfrac32; -\tfrac{x^{2}}{\nu}\bigr)}
                            {\sqrt{\nu\pi}\,B\!\bigl(\tfrac{\nu}{2},\tfrac12\bigr)}$
\quad(symmetric) \\[12pt]

Cauchy$(x_{0},\gamma)$    & $(-\infty,\infty)$ &
$ \dfrac{1}{\pi\gamma\,[1+((x-x_{0})/\gamma)^{2}]}$ &
$ \dfrac{1}{\pi}\arctan\!\Bigl(\dfrac{x-x_{0}}{\gamma}\Bigr)+\dfrac12$ \\[9pt]

Laplace$(\mu,b)$          & $(-\infty,\infty)$ &
$ \dfrac{1}{2b}\exp\!\bigl(-\tfrac{|x-\mu|}{b}\bigr)$ &
$ \begin{cases}
  \tfrac12\exp\!\bigl(\tfrac{x-\mu}{b}\bigr), & x<\mu,\\[3pt]
  1-\tfrac12\exp\!\bigl(-\tfrac{x-\mu}{b}\bigr), & x\ge\mu,
  \end{cases}$ \\[12pt]

Weibull$(k,\lambda)$      & $(0,\infty)$ &
$ \dfrac{k}{\lambda}\Bigl(\dfrac{x}{\lambda}\Bigr)^{k-1}
  e^{-(x/\lambda)^{k}}$ &
$ 1-e^{-(x/\lambda)^{k}}$ \\[9pt]

Pareto$(x_{m},\alpha)$    & $(x_{m},\infty)$ &
$ \dfrac{\alpha x_{m}^{\alpha}}{x^{\alpha+1}}$ &
$ 1-\Bigl(\dfrac{x_{m}}{x}\Bigr)^{\alpha}$ \\[9pt]

Beta$(\alpha,\beta)$      & $(0,1)$ &
$ \dfrac{x^{\alpha-1}(1-x)^{\beta-1}}{\mathrm B(\alpha,\beta)}$ &
$ I_{x}(\alpha,\beta)$\quad($I_{x}$ = regularized Beta) \\[9pt]

Rayleigh$(\sigma)$        & $(0,\infty)$ &
$ \dfrac{x}{\sigma^{2}} \exp\!\bigl(-x^{2}/(2\sigma^{2})\bigr)$ &
$ 1-\exp\!\bigl(-x^{2}/(2\sigma^{2})\bigr)$ \\[6pt]

Triangular$(a,c,b)$       & $(a,b)$ &
$ \dfrac{2(x-a)}{(b-a)(c-a)}\mathbf 1_{a\le x<c}
 +\dfrac{2(b-x)}{(b-a)(b-c)}\mathbf 1_{c\le x<b}$ &
piecewise quadratic (integral of pdf) \\[6pt]
\bottomrule
\end{tabular}
\caption{Common \emph{continuous} distributions.  Special-function notation follows standard texts.}
\end{table}
\section{Useful Identities}
\subsection{Normal}
Conditional distributions for MVN. Relationship between conditional distributions (mean) and OLS. If mean 0, then $E[Y|Z] = \frac{EYZ}{EZ^2}Z $, ala $Z (Z^TZ)^{-1} Z^T Y $  
\myworries{add this}
\subsection{Cauchy}
"Sample median is a better estimator than sample mean". Sample mean is consistent, sample mean is not. 
\subsection{Exponential}
"Sum of independent exponential with the same rate parameter is Gamma".\\
If $X_i \sim Expo(\lambda)$, then:
$$\sum_{i=1}^n X_i \sim \Gamma (n, \lambda).$$

\subsection{Uniform}
$$U_{(k)} \sim Beta(k, n+1-k)$$


\end{document}